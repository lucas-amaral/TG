% !TEX TS-program = XeLaTeX
% !TEX encoding = UTF-8 Unicode

\documentclass{beamer}

\mode<presentation>
{
  \usetheme{Dresden}

  \setbeamercovered{transparent}
  \useinnertheme{circles}
  \usecolortheme{lsc}
  \beamertemplatenavigationsymbolsempty
}

\setbeamertemplate{headline}{}
\setbeamercovered{invisible}


\usepackage[brazil]{babel}
% or whatever

%\usepackage[utf8]{inputenc}

%\usefonttheme{serif}

\usepackage[T1]{fontenc}
\usepackage{fontspec}
\setmainfont{Linux Libertine G}
\setsansfont{Source Sans Pro}
\setmonofont{Source Code Pro}

\usepackage{textcomp}
\usepackage{listings}

\lstdefinelanguage{Java}{
  %keywords={typeof, new, true, false, catch, function, return, null, catch, switch, var, if, in, while, do, else, case, break},
  keywords={struct, fn, let, box, mut, pub, impl, for, match, const},
  %keywordstyle=\color{blue}\bfseries,
  %ndkeywords={class, export, boolean, throw, implements, import, this},
  %ndkeywords={Add, Num},
  %ndkeywordstyle=\color{darkgray}\bfseries,
  %identifierstyle=\color{black},
  sensitive=true,
  comment=[l]{//},
  morecomment=[s]{/*}{*/},
  %commentstyle=\color{purple}\ttfamily,
  stringstyle=\color[rgb]{0.0,0.4,0.65}\ttfamily,
  %morestring=[b]',
  morestring=[b]"
}

\lstdefinestyle{yuriks}{
	basicstyle=\scriptsize\ttfamily,
	commentstyle=\color[rgb]{0,0.5,0}\normalfont,
}

\lstset{
	language=Rust,
	style=yuriks,
	frame=single,
	texcl=true,
	numbers=left,
	showstringspaces=false,
	}

\def\Cpp{{C\nolinebreak[4]\raisebox{.2ex}{\scriptsize\bf++}}}
%\def\Cpp{{C\nolinebreak[4]\raisebox{.3ex}{\small++}}}

\title{Desenvolvimento e Reutilização de Testes Automatizados em Aplicações Web}
\author[Lucas Antunes Amaral]{Lucas Antunes Amaral\\Orientador: Profª Drª Andrea Scwertner Charão}
\institute[UFSM]{Ciência da Computação\\Universidade Federal de Santa Maria}
\date{16/10/2015}

\pgfdeclareimage[height=0.75cm]{university-logo}{images/CienciaDaComputacao.png}
\logo{\pgfuseimage{university-logo}}

% Delete this, if you do not want the table of contents to pop up at
% the beginning of each subsection:
\AtBeginSubsection[]
{
  \begin{frame}<beamer>{Outline}
    \tableofcontents[currentsection,currentsubsection]
  \end{frame}
}


% If you wish to uncover everything in a step-wise fashion, uncomment
% the following command:

%\beamerdefaultoverlayspecification{<+->}


\begin{document}

\begin{frame}
	\titlepage
	%\pgfuseimage{university-logo}
\end{frame}

\begin{frame}{Outline}
  \tableofcontents
\end{frame}

\section{Introdução}
\begin{frame}{Introdução}
	\begin{itemize}
		\item Ferramenta utilizadas:
		\begin{itemize}
			\item Selenium
			\item JUnit
		\end{itemize}
		\item Área de aplicação: Sistemas Web
		\begin{itemize}
			\item Requisitos do sistema
		\end{itemize}
	\end{itemize}
\end{frame}

\subsection{Objetivos}
\begin{frame}{Objetivos}
	\begin{itemize}
		\item Apresentar \emph{scripts} reutilizáveis de teste
		\begin{itemize}
			\item<2-> Importar classes genéricas
			\item<2-> Pouca manutenção/implementação dos códigos
		\end{itemize}
	\end{itemize}
\end{frame}

\subsection{Justificativa}
\begin{frame}{Justificativa}
	\begin{itemize}
		\item Reduzir o trabalho na criação de novos códigos de teste
		\item Reutilizar ao máximo solução prontas para testes
		\item Aumentar qualidade da solução entregue
	\end{itemize}
\end{frame}

\section{Fundamentação}
\subsection{Teste de software}
\subsection{Ferramentas de teste de software}
\begin{frame}{Ferramentas de teste de software}
	\begin{itemize}
		\item Selenium HQ
		\item Cucumber
		\item JUnit
	\end{itemize}
\end{frame}
\section{Desenvolvimento}
\subsection{Delimitação do escopo}
\begin{frame}{Delimitação do escopo}
	\begin{itemize}
		\item Sistemas web
		\begin{itemize}
			\item Desenvolvido em Java.
			\item Suporte Ajax e Javascript.
		\end{itemize}
	\end{itemize}
\end{frame}
\pgfdeclareimage[height=4.0cm]{solucao1}{images/solucao1.png}
\subsection{Visão geral da solução}
\begin{frame}{Visão geral da solução}
    \begin{itemize}
		\item Interface de anotação 
		\item Classe de teste (Junit + Selenium) 
	\end{itemize}
	\visible<2->{\begin{center}\pgfuseimage{solucao1}\end{center}}
\end{frame}
\subsection{Discussão sobre a solução}
\section{Próximos Passos}
\subsection{Permitir testes funcionais}
\begin{frame}{Permitir testes funcionais}
	\begin{itemize}
		\item Alterar a modelagem visando permitir a realização de testes funcionais.
		\begin{itemize}
			\item Para permitir que cenários sejam contemplados por testes funcionais, será acoplado a solução a utilização da linguagem Cucumber, onde cada cenário será descrito por um conjunto de ações do Selenium.
		\end{itemize}
	\end{itemize}
\end{frame}

\subsection{Ampliar área de abrangência da solução}
\begin{frame}{Ampliar área de abrangência da solução}
	\begin{itemize}
		\item Implementar e disponibilizar suporte para elementos mais complexos.
	\end{itemize}
\end{frame}

\subsection{Validação da solução}
\begin{frame}{Validação da solução desenvolvida}
	\begin{itemize}
		\item Realizar comparações para validar solução genérica desenvolvida.
		\item Apresentar motivos pelo qual deve-se utilizar a solução e o ganho que a mesma trás para uma equipe que possui
	recursos humanos limitado.
	\end{itemize}
\end{frame}


\begin{frame}
\titlepage
\end{frame}

\end{document}