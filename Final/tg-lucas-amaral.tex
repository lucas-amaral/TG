% !TEX TS-program = pdfLaTeX+MakeIndex+BibTeX
% !TEX encoding = UTF-8 Unicode

\PassOptionsToPackage{unicode}{hyperref}
\PassOptionsToPackage{naturalnames}{hyperref}

\documentclass[tg]{mdtufsm}

\usepackage[T1]{fontenc}
\usepackage{fix-cm}
\usepackage{microtype} % Nicer text flowing & typography
\usepackage{times, color}
\usepackage[utf8]{inputenc}
\usepackage{graphicx}
\usepackage{amsmath,latexsym,amssymb}
%\usepackage[hidelinks]{hyperref}
\usepackage[hidelinks,
            bookmarksopen=true,linktoc=none,colorlinks=true,
            linkcolor=black,citecolor=black,filecolor=magenta,urlcolor=blue,
            pdftitle={Desenvolvimento e Reutilização de Testes Automatizados em Aplicações Web},
            pdfauthor={Lucas Antunes Amaral},
            pdfsubject={Trabalho de Graduação},
            pdfkeywords={Qualidade de Software, Testes automatizados de Software, Linguagens de Programação, Selenium HQ, Cucumber, Informática, UFSM}
            ]{hyperref}
%\usepackage[brazilian]{babel}

%\usepackage{fontspec}
%\setmainfont{Linux Libertine G}

%%% PAGE DIMENSIONS
\usepackage[inner=30mm,outer=20mm,top=30mm,bottom=20mm]{geometry}
%\usepackage{epstopdf}
\usepackage{graphicx}
\usepackage{pdfpages}
% \geometry{margin=2in} % for example, change the margins to 2 inches all round
% \geometry{landscape} % set up the page for landscape

% \usepackage[parfill]{parskip} % Activate to begin paragraphs with an empty line rather than an indent

%%% PACKAGES
%\usepackage{amsfonts}
%\usepackage{color}
%\usepackage{booktabs} % for much better looking tables
%\usepackage{array} % for better arrays (eg matrices) in maths
%\usepackage{paralist} % very flexible & customisable lists (eg. enumerate/itemize, etc.)
\usepackage{siunitx}
%\usepackage{verbatim} % adds environment for commenting out blocks of text & for better verbatim
\usepackage{listings}
\usepackage{subcaption}
\captionsetup{compatibility=false}
%\usepackage{microtype}
%\usepackage[numbers]{natbib}
%\usepackage{subfig} % make it possible to include more than one captioned figure/table in a single float
% These packages are all incorporated in the memoir class to one degree or another...

\lstdefinelanguage{Java}{
  keywords={public, private, new, true, false, catch, void, return, null, for, switch, var, if, in, while, else, case, break},
  %keywords={struct, fn, let, box, mut, pub, impl, for, match, const},
  keywordstyle=\color[rgb]{0.4,0.4,0.65}\bfseries,
  ndkeywords={class, export, boolean, throw, implements, import, this,
    @Before, @Test, @After, Funcionalidade, Contexto, Dado, Quando, E, Entao, @Dado, @Quando, @E, @Entao, @Documented, @Retention, @Target, @Teste},
  %ndkeywords={Add, Num},
  ndkeywordstyle=\color{purple}\bfseries,
  %identifierstyle=\color{black},
  sensitive=true,
  comment=[l]{//},
  morecomment=[s]{/*}{*/},
  commentstyle=\color{purple}\ttfamily,
  stringstyle=\color[rgb]{0.0,0.4,0.65}\ttfamily,
  %morestring=[b]',
  morestring=[b]"
}

\lstset{
	basicstyle=\scriptsize\ttfamily,
	tabsize=2,
	frame=single,
	breaklines=true,
	breakatwhitespace=true,
	xleftmargin=0cm,
	xrightmargin=0cm,
	literate=
		{á}{{\'a}}1 {é}{{\'e}}1 {í}{{\'i}}1 {ó}{{\'o}}1 {ú}{{\'u}}1
		{Á}{{\'A}}1 {É}{{\'E}}1 {Í}{{\'I}}1 {Ó}{{\'O}}1 {Ú}{{\'U}}1
		{à}{{\`a}}1 {è}{{\`e}}1 {ì}{{\`i}}1 {ò}{{\`o}}1 {ù}{{\`u}}1
		{À}{{\`A}}1 {È}{{\'E}}1 {Ì}{{\`I}}1 {Ò}{{\`O}}1 {Ù}{{\`U}}1
		{ä}{{\"a}}1 {ë}{{\"e}}1 {ï}{{\"i}}1 {ö}{{\"o}}1 {ü}{{\"u}}1
		{Ä}{{\"A}}1 {Ë}{{\"E}}1 {Ï}{{\"I}}1 {Ö}{{\"O}}1 {Ü}{{\"U}}1
		{â}{{\^a}}1 {ê}{{\^e}}1 {î}{{\^i}}1 {ô}{{\^o}}1 {û}{{\^u}}1
		{Â}{{\^A}}1 {Ê}{{\^E}}1 {Î}{{\^I}}1 {Ô}{{\^O}}1 {Û}{{\^U}}1
		{ã}{{\~a}}1 {Ã}{{\~A}}1 {õ}{{\~o}}1 {Õ}{{\~O}}1
		{ç}{{\c c}}1 {Ç}{{\c C}}1,
	texcl=true,
	%numbers=left,
	showstringspaces=false,
	commentstyle=\normalfont
}

% For Computer Modern:
%\def\Cpp{{C\nolinebreak[4]\hspace{-.05em}\raisebox{.4ex}{\tiny\bf ++}}}
% For Linux Libertine G
\def\Cpp{{C\nolinebreak[4]\raisebox{.20ex}{\small\bf++}}}

%\newcommand{\todo}[1]{\textsf{\color{red}#1}}
\newcommand{\todo}[1]{}
\graphicspath{{./images/}}

\input{macros/bugcaption}

%%% END Article customizations

\title{Desenvolvimento e Reutilização de Testes Automatizados em Aplicações Web}
\author{Amaral}{Lucas Antunes}
\course{Curso de Ciência da Computação}
\altcourse{Curso de Ciência da Computação}
\institute{Centro de Tecnologia}
\degree{Bacharel em Ciência da Computação}

\trabalhoNumero{}
\advisor[Profª.]{Drª.}{Charão}{Andrea Schwertner}
\orientadoratrue

\committee[Profª. Dr.]{Bernardi}{Giliane}{UFSM}
\committee[MSc.]{Pereira}{Adriano}{UFSM}

\date{4}{Dezembro}{2015}

\keyword{Qualidade de Software}
\keyword{Testes automatizados de Software}
\keyword{Linguagens de Programação}
\keyword{Selenium HQ}
\keyword{Cucumber}

%\date{} % Activate to display a given date or no date (if empty), otherwise the current date is printed

\begin{document}
\maketitle
%\makeapprove
%\includepdf[pages={-}]{folha_de_aprovacao.pdf}

\chapter*{Agradecimentos}
\begin{abstract}
A constante busca pela qualidade de uma solução em forma de software, fez com que empresas do ramo de desenvolvimento aderissem a realização de testes automatizados em seus sistemas. A partir deste cenário,
surgiram inúmeras ferramentas e \emph{frameworks} para suprir esta demanda, que se propõem  a ampliar a otimização de tempo e
eficácia das aplicações implementadas, visando uma garantia maior na qualidade das mesmas. Contudo, é sabido que criar
um novo teste para cada nova funcionalidade ou demanda do sistema, torna-se muito custoso, sendo necessário um grande
desprendimento de recursos humanos. Assim, este trabalho objetiva apresentar \emph{scripts} de testes automatizados para
sistemas web, que possam, de maneira mais genérica e com poucas alterações, serem reutilizados, de forma escalável,
para novos casos de testes, que sigam o mesmo escopo.
\end{abstract}

%\begin{englishabstract}
%{}
%{}
%{}
%{December}
%\end{englishabstract}

\tableofcontents
\listoffigures
\listoftables
%\listofappendix

\setlength{\baselineskip}{1.5\baselineskip}

%	\item[Período de execução:] Setembro de 2014 a Dezembro de 2014
%	\item[Unidades participantes:] ~\\ Curso de Ciência da Computação \\ Departamento de Eletrônica e Computação
%	\item[Área de conhecimento:] Ciência da Computação
%	\item[Linha de Pesquisa:] Computação Gráfica, Linguagens de Programação, Programação Paralela
%	\item[Tipo de projeto:] Trabalho de Conclusão de Curso

\chapter{Introdução}

Dada a atual conjectura do mercado de desenvolvimento de software é fundamental, para
que uma aplicação se mantenha viva de forma competitiva, apresentar diferenciais ao seu público alvo. Desta maneira, a área de qualidade de software ganha cada vez mais espaço dentro das empresas de tecnologia da informação e, em especial, as ferramentas e
metodologias de teste de software ganham maior visibilidade. Além do aumento da produtividade e da diminuição do retrabalho, as empresas na área de tecnologia da informação buscam um melhor relacionamento com seus clientes por meio do melhor planejamento e gestão de suas atividades de desenvolvimento e da diminuição no número de defeitos nos produtos entregues \cite{jomori2004qualidade}.

Os testes de software podem ocorrer em todas as etapas do desenvolvimento e de diferentes formas, contudo,
sempre objetivam atender na totalidade os requisitos do sistema e, simultaneamente, amplificar a qualidade da solução
codificada. Apesar de, aparentemente, ser mais prático e rápido realizar um teste manual, a cada nova alteração em um módulo do sistema,
o teste tem que ser todo refeito e a tendência que novos erros sejam gerados, até mesmo em funcionalidades já testadas, é enorme,
problema este que não ocorre quando a abordagem escolhida é a automatização dos testes. 

Os testes automatizados angariam cada vez mais adeptos ao longo dos últimos anos. Tal fato se deve principalmente à grande redução de custo observada a médio e a longo prazos com o uso desta prática \cite{de2007utilizaccao}. Em contrapartida, os testes automatizados demandam um grande custo inicial em sua codificação e, com isso, aumentam o
envolvimento da equipe de qualidade. Pensando nesse problema, podemos buscar formas alternativas para que se possa usufruir de todas estas virtudes dos testes automatizados e, ao mesmo tempo, utilizar de forma eficiente os recursos disponíveis em uma instituição.

\section{Objetivos}

\subsection{Objetivo Geral}

Este trabalho tem como objetivo principal apresentar um conjunto de classes de teste e testes que possam ser reutilizados de forma otimizada em novas funcionalidades de uma aplicação ou em sistemas que sigam os mesmos padrões e comportamento dos softwares para os quais foram desenvolvidos.

\subsection{Passos de metodologia}
\begin{itemize}
	\item Apresentar solução para testes unitários utilizando interface de anotação + Selenium WebDriver;
    \item Exibir solução para testes funcionais utilizando Cucumber + Selenium WebDriver;
    \item Apresentar solução para testes funcionais utilizando Selenium WebDriver;
    \item Expor qualidades e defeitos das soluções apresentadas.
\end{itemize}

\section{Justificativa}

A qualidade de software  é uma das variáveis essenciais para que um projeto de software tenha sucesso.
Sendo assim, torna-se cada vez mais necessária a inserção de testes automatizados em projetos web agregando, aos mesmos, uma maior confiabilidade e redução nos possíveis erros que o sistema possa
apresentar. Para que haja a possibilidade de aumentar a qualidade dos sistemas, sem que seja necessária uma maior
demanda de recursos humanos para a área de qualidade, podemos adotar práticas de reuso de códigos de testes, visando
maximizar a produtividade e eficiência, além de, simultaneamente, obter um produto final com uma garantia de qualidade
superior.

\chapter{Fundamentos e Revisão de Literatura}

Neste capítulo, serão apresentados conceitos relativos aos conteúdos abordados neste trabalho, descrevendo qualidade de software, ferramentas de teste de software, assim como, o reuso de testes.

\section{Qualidade de Software}

A Qualidade de software é uma subárea, oriunda da engenharia de software, que tem como foco central apresentar metodologias,
procedimentos e métricas que garantam a qualidade no processo de desenvolvimento de um sistema. Apesar de ocorrer no processo, a qualidade de software
objetiva obter qualidade no produto final e, com isso, conseguir contemplar na totalidade os requisitos tratados com o cliente ao longo do processo.
\citeauthor{de2006introduccao} \cite{de2006introduccao} afirmam que a qualidade de software está diretamente relacionada a um gerenciamento
rigoroso de requisitos, uma gerência efetiva de projetos e em um processo de desenvolvimento bem definido gerenciado e em melhoria contínua. Afirmam também,
que atividades de verificação e uso de métricas para controle de projetos e processo também estão inseridas nesse contexto, contribuindo para tomadas de
decisão e para antecipação de problemas.

Mas como medir a qualidade de um sistema em questão? Para responder este questionamento, Garvin \cite{garvin1987competing} propõe o conceito que ele chama de
oito dimensões que seriam, em ordem, qualidade do desempenho, qualidade dos recursos, confiabilidade, conformidade, durabilidade, facilidade de manutenção,
estética e percepção. Segundo \citeauthor{garvin1987competing}, atendendo a estes oito critérios, o sistema apresentará qualidade.
\citeauthor{pressman2011engenharia} \cite{pressman2011engenharia} complementa a definição de Garvin mesclando-a com a norma ISO 9126, e aponta os fatores
críticos para o sucesso neste caso, como sendo: Intuição, Eficiência, Robustez e Riqueza.

\subsection{Qualidade do processo}
A qualidade de software é largamente determinada pela qualidade dos processos utilizados para o desenvolvimento. Deste modo, a melhoria
da qualidade de software é obtida pela melhoria da qualidade dos processos \cite{koscianski2007qualidade}.

\subsection{Qualidade do produto}
Existe uma relação direta entre qualidade de produto e qualidade do processo, pois, para obtenção da qualidade do produto final,
faz-se necessário adquirir primeiramente qualidade nos processos que compõem o desenvolvimento do mesmo.
Avaliar a qualidade de um produto de software é verificar, através de técnicas e atividades operacionais, o quanto os requisitos são atendidos. Tais requisitos,
de uma maneira geral são a expressão das necessidades, explicitados em termos quantitativos ou qualitativos e têm por objetivo definir as características de
um software, a fim de permitir o exame de seu atendimento \cite{koscianski2007qualidade}.

\section{Testes de Software}
É a atividade responsável por apresentar os erros existentes em um determinado programa. Por isso, pode ser vista como uma atividade destrutível, pois visa expor os defeitos para depois corrigir os mesmos e, de preferência, em
um estágio inicial. Quanto mais tarde um defeito for identificado, mais caro fica para corrigi-lo, e mais, os custos de descobrir e corrigir o defeito no software aumentam exponencialmente na proporção em que o trabalho
evolui através das fases do projeto de desenvolvimento \cite{boehm1976quantitative}. O teste possibilita também validar se os requisitos inicias do sistema, alinhados pelos \emph{stakeholders}, estão contemplados em sua plenitude.

Apesar de não ser possível, através de testes, provar que um programa está correto, os testes, se conduzidos sistemática e criteriosamente, contribuem para
aumentar a confiança de que o software desempenha as funções especificadas e evidenciar algumas características mínimas do ponto de vista da qualidade do produto \cite{maldonado2004introduccao}. Sendo assim,
faz-se essencial o mapeamento de um processo de testes para que se possa criar garantias e métricas que reduzam os erros, maximizando a qualidade, \citeauthor{crespo2004metodologia} \cite{crespo2004metodologia} descrevem o processo de
teste como sendo a composição de quatro macro etapas: Planejamento, projeto, execução e acompanhamento dos testes de unidade.

\subsection{Plano de teste}
Embora pareça ideal realizar testes sobre o projeto inteiro, muitas vezes esta tarefa pode torna-se inviável devido ao tamanho e complexidade do sistema em questão. Desta forma, torna-se necessário descrever as áreas e testes necessários em um sistema. Podemos retratar esta situação com um plano de teste.
O Plano  de  Teste estrutura  e  organiza  as informações  referentes  ao  processo  de  teste  durante  todo  o  projeto,  facilitando assim  o  planejamento  do  
Projeto  de  Teste \cite{piqueiro2015}. Pode ser visto como uma etapa de planejamento do projeto e pode ser descrito como um conjunto de casos de teste.

\subsection{Caso de teste}
O propósito do Caso de Teste é definir uma unidade de teste que será executada pelo testador, seja manual ou automaticamente \cite{rios2006teste}. Um caso de teste é constituído visando descrever passos sequenciais necessários para validação de um artefato de teste. O caso de teste deve especificar a saída esperada e os resultados esperados do processamento \cite{myers2011art}.

\subsection{Tipos de Teste}
Existem muitos tipos de teste, mas podemos destacar entre elas: teste de unidade, integração, sistema, aceitação e regressão. Serão descritos nas próximas sub-seções os tipos utilizados no desenvolvimento deste trabalho.

\subsubsection{Teste de unidade}
O teste de unidade focaliza o esforço de verificação na menor unidade de projeto do software \cite{pressman2011engenharia}. Em uma aplicação orientada a objeto o teste neste caso seria voltado ao método.

\subsubsection{Teste de aceitação}
Um teste de aceitação individual tem o objetivo de verificar se um critério de aceitação de uma história ou um requisito foi atendido \cite{dornbusch2014entrega}.
 
\subsection{Estratégias de Testes}
A Estratégia de testes se caracteriza pela definição da abordagem geral a ser aplicada nos testes, descrevendo como o software será testado, identificando os níveis de testes que serão aplicados, os métodos,
técnicas e ferramentas a serem utilizada \cite{rios2006teste}. Deve-se também escolher na etapa de estratégia os tipos de testes que deverão ser realizados assim como os estágios de teste escolhidos. Uma estratégia de teste de software deve acomodar testes de baixo nível, necessários para verificar se um pequeno segmento de código fonte foi implementado corretamente, bem com testes de alto nível, que validam as funções principais do sistema de acordo com os requisitos do cliente \cite{pressman2011engenharia}.

Tradicionalmente entre desenvolvedores de teste, utiliza-se uma estratégia onde temos um equilíbrio entre testar o sistema após pronto e testar somente antes de desenvolver o sistema. Esta estratégia assume uma visão incremental do teste, começando com o teste de unidades individuais de programa, passando para os testes destinados a facilitar a integração de unidades e culminando com testes que usam o sistema concluído \cite{pressman2011engenharia}.

\subsection{Testes de Sistemas Web}
O teste de aplicações Web se reveste de características peculiares, uma vez que se tratam, em geral, de aplicações colocadas na Internet e cujo acesso, em teoria, é aberto a qualquer usuário \cite{rios2006teste}. Sendo assim o tais sistemas diferem-se das aplicações web pelos cuidados necessários com questões como a possibilidade de utilização de sistemas operacionais e navegadores distintos, segurança e conectividade. Além
disso, a relevância econômica das aplicações Web aumenta a importância de controlar e de
melhorar a sua qualidade \cite{fidelis2004fireweb}.


\subsection{Ferramentas para teste de software}
Existem muitas ferramentas desenvolvidas para realização de testes de software web. Neste trabalho serão descritas três(3) dentre elas, que serão as ferramentas utilizadas no desenvolvimento dos testes estudados.

\subsubsection{Cucumber}
Cucumber é um ferramenta de desenvolvimento de testes, voltado para sistemas web, que adota uma linguagem de alto nível bem próxima a uma linguagem natural e tem suas origens fixadas sobre a metodologia BDD (\emph{Behavior Driven Development}). Cucumber
é escrita em linguagem Ruby, mas pode ser utilizada para executar especificações de aplicações escritas em qualquer linguagem \cite{nunescucumber}.

A escolha dessa ferramenta baseou-se na fácil transcrição dos requisitos do sistema para a linguagem em questão, tornando possível conferir se os requisitos estão contemplados pelas funções e métodos descritos no sistema web.
\citeauthor{lopescucumbervalor}\cite{lopescucumbervalor} compara Cucumber com o software Capybara\footnote{Capybara é uma biblioteca escrita na linguagem de programação Ruby que torna fácil simular como um usuário interage com sua aplicação. http://jnicklas.github.io/capybara/} e afirma que o primeiro apresenta um código mais legível e amigável. Uma observação é que o código com Capybara faz
referência para vários detalhes de implementação, enquanto o código do Cucumber reserva isso apenas para os \emph{Steps} e não para o arquivo de \texttt{feature} \cite{lopescucumbervalor}.

A ferramenta funciona basicamente através da leitura de arquivos com a extensão \texttt{feature}, os quais descrevem em linguagem natural uma funcionalidade e casos de teste, conhecidos como cenários.
Como os testes estão escritos em uma linguagem natural e não de programação, Cucumber precisa pesquisar pelo código associado aos passos que formam o cenário em arquivos auxiliares \cite{scmitzcucumberreview}. Cucumber
executa seus arquivos \texttt{.feature} e esses arquivos contêm especificações executáveis escritos em uma linguagem chamada Gherkin \cite{cucumberwiki}, que possui um layout bem definido. Inicia pela descrição de uma funcionalidade
que, por sua vez, possui cenários; onde, um cenário é descrito da seguinte forma: \texttt{Dado} alguma condição \texttt{Quando} outra condição \texttt{E} terceira condição \texttt{Então} faça algo, conforme ilustra a figura \ref{fig:codigo_cucumber}.

\begin{figure}[!htb]
    \centering
    \includegraphics[width=1.1
    \textwidth]{codigo_cucumber}
    \caption{Um exemplo de código cucumber.}
    \label{fig:codigo_cucumber}
\end{figure}

\subsubsection{Selenium HQ}
É um \emph{framework} open source utilizado para automatização de testes funcionais em aplicações web \cite{chiavegatto1desenvolvimento}. Segundo \citeauthor{pereiraestudoselenium} \cite{pereiraestudoselenium} se trata
de uma ferramenta de fácil uso e eficiente para desenvolver casos de teste, permitindo os testes de aceitação ou funcional, regressão e de desempenho.
Selenium trabalha como um plugin do navegador Firefox, o mesmo traz muita praticidade pois permite que se possa capturar cliques e valores digitados, transformando-os em um caso de teste. Ele é composto por
quatro ferramentas: Selenium IDE, Selenium Grid, Selenium RC e Selenium WebDrive.

A utilização de comandos no Selenium consiste em digitar o comando seguido de dois parâmetros tal como, por exemplo, \texttt{verifyText //div//a[2] Login}. Dependendo  do comando os parâmetros poderão ser opcionais, pois, alguns comandos não necessitam de parâmetros para serem executados \cite{sixpenceautomatizaccao}.

Selenium foi escolhida como umas das ferramentas no desenvolvimento deste projeto pelo fato de ser um \emph{framework open source} que possui um vasto leque de ferramentas para testes de sistemas web. Outro fator
importante para sua escolha é a possibilidade de interação do Selenium HQ com o Cucumber. Ela se destaca entre as demais ferramentas gratuitas pelo fato de ser a mais completa, permitindo integração com
várias linguagens, outros \emph{frameworks}, além de suportar inúmeros navegadores e sistemas operacionais \cite{pereiraestudoselenium}.

\subsubsection{JUnit}
JUnit é uma ferramenta de apoio ao teste unitário, a qual auxilia desenvolvedores na automação dos testes e verificação dos resultados \cite{biasi2006geraccao}. A escolha desta ferramenta, como parte integrante da solução desenvolvida, deveu-se ao fato de a mesma ser voltada para sistemas desenvolvidos na linguagem de programação Java, além de ser altamente versátil, possibilitando assim, integração em um único código da mesma com os demais \emph{frameworks} que serão utilizados(Selenium e Cucumber).

Justifica-se a sua utilização neste trabalho por tratar-se de uma API (\emph{Application Programming Interface}), que viabiliza a comparação de um valor obtido em algum teste com o valor esperado pelo mesmo. Desta forma, pode-se validar e verificar se
a funcionalidade de um sistema está trabalhando adequadamente.


\section{Reuso de testes}

Visando melhor aproveitar os recursos existentes em uma instituição e, ainda assim apresentar garantias na qualidade do produto/serviço entregues aos clientes, desenvolvedores do mundo todo começaram a apresentar
teses e modelos que buscam criar testes genéricos e padronizados. Um padrão é um pedaço de informação instrutiva e nomeada, que captura a estrutura essencial
e \emph{?insights?}, de uma família bem sucedida de soluções aprovadas, para um determinado problema, o qual surge em um determinado contexto \cite{cagnin2004reuso}.

\citeauthor{guizzardi2000desenvolvimento} diz que, por razões históricas, a área de desenvolvimento de software não atingiu a maturidade que outras áreas da engenharia atingiram. Complementa afirmando que, apesar disso, é inegável que algum avanço tenha sido alcançado, pois a forma de realização dessa atividade evoluiu de uma atividade realizada de forma quase artesanal, para um processo de
desenvolvimento bem estruturado e que, nos melhores casos, contempla inclusive atividades de gerência e avaliação da qualidade \cite{guizzardi2000desenvolvimento}.
Com isso, a reutilização dos testes já codificados se mostra uma importante prática no desenvolvimento e que ainda apresenta muitas incógnitas e possibilidades para as equipes de TI,
principalmente na geração de casos de testes que possam ser reutilizados em situações que apresentem um padrão parecido com os casos já conhecidos.

Como solução em forma de reutilização de teste, \cite{cagnin2004reuso} apresentam uma abordagem composta por: a) uma estratégia, que define e associa requisitos de teste a padrões de linguagens de padrões  de análise; b) diretrizes, que apoiam o engenheiro de software na decisão de quais requisitos de teste disponíveis devem ser reusados e instanciados para casos de teste concretos. A estratégia que define os requisitos de teste da abordagem proposta foi aplicada aos padrões de uma
linguagem de padrões de análise, utilizados na reengenharia de um sistema legado de biblioteca. Já \cite{karinsalo2004software} sugerem um modelo de processo de desenvolvimento de teste que leva técnicas de reutilização de software e atividades em conta, revela ainda que, a fim de produzir material de teste reutilizável, as entidades de software tem de ser expressa em termos de recursos, em que os materiais de ensaio estão anexados.

Segundo \cite{patel2014test}, Tata Consultancy Services (TCS)  é uma empresa provedora de serviços em TI (ITSP) que desenvolveu um repositório de casos de teste reutilizáveis com o objetivo de reduzir o custo dos testes em projetos de Implementação de software empresarial (ESI).\cite{patel2014test} trás um estudo do programa de reutilização para analisar o seu impacto. Um conjunto de métricas foram definidos e os dados pertinentes foram obtidas a partir de usuários do repositório.

\cite{crespo2004metodologia} apresentam uma metodologia para implantação ou melhoria do processo de teste em empresas desenvolvedoras de software, com o objetivo de viabilizar a utilização das práticas de teste pelas empresas. Apesar de não tratar-se de uma solução que contenha reutilização de software este trabalho tem relação direta com o trabalho desenvolvido pois a reutilização nada mais visa que melhorias no processo de desenvolvimento dos testes de software.

\chapter{Desenvolvimento}

A seguir serão descritas as atividades desenvolvidas a fim de alcançar os objetivos propostos por este trabalho. Será apresentado o escopo do projeto, assim como trechos dos códigos desenvolvidos e estrutura das soluções encontradas ao longo do desenvolvimento do projeto.

\section{Delimitação de escopo}

Como escopo, foram delimitados alguns pré-requisitos necessários para utilização da solução proposta neste trabalho, onde o sistema deve ser um software web desenvolvido na linguagem de programação Java,
podendo apresentar também funções Ajax e Javascript.

\section{Visão geral da solução}
Em um primeiro momento, com objetivo de automatizar os testes de uma aplicação a fim de reduzir o esforço na confecção e codificação de novos testes para as equipes de desenvolvimento e testes de uma organização,
escolheu-se desenvolver uma solução baseada em três etapas que são descritas conforme imagem \ref{fig:solucao1}.

\begin{figure}[!htb]
	\centering
	\includegraphics[width=0.8\textwidth]{solucao1}
	\caption{Estrutura da solução.}
	\label{fig:solucao1}
\end{figure}

Temos como ponto de partida um sistema onde se possui pouco ou nenhum teste codificado. Para o mesmo descrevemos  três fases no desenvolvimento de uma solução, em forma de testes para o sistema,
que seriam: criar uma classe de testes genérica, responsável por realizar os testes necessários na aplicação, descrever uma interface de anotação que representaria ações de elementos web (\emph{HTML}) que
seriam testados posteriormente, e por fim, inserir as anotações nos métodos e classes do sistema que devem ser testados nesta aplicação.

\section{Classe genérica de testes}
Por tratar-se de uma abordagem para sistemas web, optou-se pela utilização de dois frameworks de testes, que possibilitam interações com elementos web, para confecção da classe responsável pela execução dos testes automatizados.
O primeiro é o Selenium. Através da utilização da classe \texttt{WebDrive}, contida em seu pacote, podemos descrever uma sequência de passos que são executados por um navegador, como cliques e inserção de dados em campos de um
formulário. Sendo assim, instanciando um objeto \texttt{WebDrive}, podemos iniciar um navegador e realizar uma série de instruções pré-determinadas para validar se a aplicação comporta-se de forma adequada.

Por fim, necessitamos validar se após realizar o preenchimento de formulários do software, o resultado obtido condiz com o resultado esperado. Neste momento é onde se faz necessária a adição de trechos de códigos Junit na classe de teste
desenvolvida, pois o mesmo disponibiliza em seu pacote um conjunto de métodos \emph{assert}, com os quais é possível comparar, por exemplo, se após salvar um formulário de cadastro, a mensagem de "cadastrado com sucesso" aparece na tela.
A imagem \ref{code:TestaFormularios.java} apresenta um trecho da classe confeccionada para realização dos testes.

\begin{figure}[!htb]
\begin{lstlisting}
@Test
public void testaFormularios() {
	for (Class classe : getCarregaClasses()) {
    		Teste testeClasse = TestePropriedades.teste(classe);
        if (testeClasse.fazerLogin()) {
            LoginTeste.login(TestePropriedades.urlSistema, testeClasse.getSenha(), testeClasse.getLogin(), webDriver);
        }
        if (!testeClasse.getUrl().equals("")) {
            webDriver.get(TestePropriedades.urlSistema + testeClasse.getUrl());
        }
        for(Method metodo: classe.getDeclaredMethods()) {
            Teste teste = TestePropriedades.teste(metodo);
            if (teste != null) {
                executaTeste(teste, true);
            }
        }
        executaTeste(testeClasse, false);
        System.out.println("Formulário da classe " + classe.getName() + " testado!");
	}
}
\end{lstlisting}
    \caption{Classe genérica de teste desenvolvida}
	\label{code:TestaFormularios.java}
\end{figure}

A classe desenvolvida possui um método principal denominado \texttt{testaFormularios}, onde a ideia principal do mesmo seria buscar,
de forma recursiva, todas as classes do projeto anotadas pela interface que indica que a classe deve ser testada e para cada uma das mesmas cria-se elementos \texttt{webDrive} para executar as ações indicadas pela anotação.

\section{Interface de anotação}

Após finalizar o desenvolvimento da classe de testes genérica, passou a existir a necessidade de criar um mecanismo para informar, para a classe de testes, quais seriam as validações necessárias. Para esta tarefa,
por tratar-se de projetos Java, resolveu-se criar uma interface de anotação que seria vinculada aos métodos e classes, onde seriam necessários executar os testes posteriores. Desta forma, a classe \texttt{TestaFormulario.java},
responsável pela execução dos testes, varreria as classes do sistemas, e toda vez que encontrar uma classe anotada por essa interface, executaria um teste, onde, cada método que necessitaria de um teste possuiria em sua anotação o tipo do campo,
as ações que deveriam ser executas e o valor que deveria ser preenchido.

Deu-se o nome de \texttt{Teste.java} para a interface, e sua estrutura está ilustrada em \ref{code:Teste.java}.

\begin{figure}[!htt]
\begin{lstlisting}
@Documented
@Retention(RetentionPolicy.RUNTIME)
@Target({ElementType.TYPE, ElementType.METHOD})
public @interface Teste {
    String getUrl() default "";

    //findElement
    String getCampo() default ""; //campo html do formulário
    String getIdentificador() default "id"; //Informar se deve buscar um id, name, class ou css

    boolean isSelect() default false;

    String getValor() default ""; //utilizado como sendKeys e selectText
    boolean click() default false;
    boolean submit() default false;
    boolean limpar() default false;

    String getTipoAssert() default "igual";
    String getCampoAssert() default "";
    String getIdentificadorAssert() default "id";
    String getValorEsperadoAssert() default "";
    String getAtributoCampoComparacaoAssert() default "texto";

    boolean fazerLogin() default false;
    String getLogin() default "colegiado";
    String getSenha()default "";
}
\end{lstlisting}
	\caption{Interface Teste}
	\label{code:Teste.java}
\end{figure}

Tem-se então que cada método existente nesta classe representa uma ação específica da classe \texttt{WebDrive}, possibilitando, assim, manipular elementos \emph{HTML} na execução do teste.

\section{Inserção das anotações no projeto}
Por fim, a ultima etapa da solução em questão, trata-se de uma etapa contínua no projeto, pois sempre que uma nova classe é mapeada no sistema ou um novo cadastro é criado, a mesma ocorrerá. Nesta etapa se faz necessária a
adição das anotações nos códigos do projeto, onde se deve inserir a anotação em todas as classes e métodos que devem ser testados.
O código \ref{code:Disciplina.java} ilustra e exemplifica uma classe Java contendo anotações referentes a interface Teste.

\begin{figure}[!htt]
\begin{lstlisting}
@Teste(getUrl = "/cadastro-disciplina.htm", getCampo = "salvar", click = true,getIdentificadorAssert = TestePropriedades.IDENTIFICADOR_CSS, getCampoAssert = "h4", getValorEsperadoAssert = "Sucesso!")
public class Disciplina {
    private String codigo;
    private String nome;
    private Integer cargaHoraria;

    @Teste(getCampo = "ativa1", click = true)
    public Boolean getAtiva() {
        return ativa == null || ativa;
    }

    public void setAtiva(Boolean ativa) {
        this.ativa = ativa;
    }

    @Teste(getCampo = "cargaHoraria", getValor = "60", isSelect = true)
    public Integer getCargaHoraria() {
        return cargaHoraria;
    }

    public void setCargaHoraria(Integer cargaHoraria) {
        this.cargaHoraria = cargaHoraria;
    }

    @Teste(getCampo = "nome", getValor = "Disciplina teste")
    public String getNome() {
        return nome;
    }
}
\end{lstlisting}
	\caption{Classe anotada com @Teste}
	\label{code:Disciplina.java}
\end{figure}

Na anotação são inseridas informações referente às ações que devem ser executadas para a propriedade em questão. Os métodos possuem anotações referentes apenas ao campo que representam, enquanto as informações mais gerais sobre
a página que está sendo testada, são inseridas na anotação da Classe propriamente dita.

\section{Discussão sobre a solução}
Após a finalização do desenvolvimento desta primeira solução encontrada, notou-se a clara necessidade de alteração na sistemática de como os teste ocorreriam. Necessidade esta oriunda de algumas limitações que a solução trás
como, por exemplo, a não possibilidade de realizar testes que envolvam mais de um método, tornando praticamente inviável a tarefa de realizar testes funcionais, pois não seria possível mapear um cenário mais completo, que depende
de mais de uma variável e tem uma sequência bem mapeada que deve ser seguida.

Outro fator determinante para a alteração desta solução foi o fato de, ao se utilizar anotações, o mesmo não possibilitaria o teste de um mesmo campo, com dois valores diferentes, pois só seria possível informar na anotação do
método um único valor utilizado no teste. Além disto, também não seria viável a execução de testes unitários para um sistema muito grande, pois é sabido que os testes devem ser executados sobre os pontos críticos do software e não na totalidade do projeto.
%\end{lstlisting}

\chapter{Resultados}

\subsection{Discussão dos Resultados}
\chapter{Conclusão}

\setlength{\baselineskip}{\baselineskip}
\bibliographystyle{abnt}
\bibliography{../referências}
\end{document}