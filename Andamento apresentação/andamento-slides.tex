% !TEX TS-program = XeLaTeX
% !TEX encoding = UTF-8 Unicode

\documentclass{beamer}

\mode<presentation>
{
  \usetheme{Dresden}

  \setbeamercovered{transparent}
  \useinnertheme{circles}
  \usecolortheme{lsc}
  \beamertemplatenavigationsymbolsempty
}

\setbeamertemplate{headline}{}
\setbeamercovered{invisible}


\usepackage[brazil]{babel}
% or whatever

%\usepackage[utf8]{inputenc}

%\usefonttheme{serif}

\usepackage[T1]{fontenc}
\usepackage{fontspec}
\setmainfont{Linux Libertine G}
\setsansfont{Source Sans Pro}
\setmonofont{Source Code Pro}

\usepackage{textcomp}
\usepackage{listings}

\lstdefinelanguage{Java}{
  keywords={public, private, new, true, false, catch, void, return, null, for, switch, var, if, in, while, else, case, break},
  %keywords={struct, fn, let, box, mut, pub, impl, for, match, const},
  keywordstyle=\color[rgb]{0.4,0.4,0.65}\bfseries,
  ndkeywords={class, export, boolean, throw, implements, import, this, 
    @Before, @Test, @After, Funcionalidade, Contexto, Dado, Quando, E, Entao, @Dado, @Quando, @E, @Entao, @Documented, @Retention, @Target, @Teste},
  %ndkeywords={Add, Num},
  ndkeywordstyle=\color{purple}\bfseries,
  %identifierstyle=\color{black},
  sensitive=true,
  comment=[l]{//},
  morecomment=[s]{/*}{*/},
  commentstyle=\color{purple}\ttfamily,
  stringstyle=\color[rgb]{0.0,0.4,0.65}\ttfamily,
  %morestring=[b]',
  morestring=[b]"
}

\lstset{
	language=Java,
	basicstyle=\tiny,
	commentstyle=\color[rgb]{0,0.2,0}\normalfont,
	frame=single,
	%texcl=true,
	numbers=left,
	showstringspaces=false,
}

\def\Cpp{{C\nolinebreak[4]\raisebox{.2ex}{\scriptsize\bf++}}}
%\def\Cpp{{C\nolinebreak[4]\raisebox{.3ex}{\small++}}}

\title{Desenvolvimento e Reutilização de Testes Automatizados em Aplicações Web}
\author[Lucas Antunes Amaral]{Lucas Antunes Amaral\\Orientador: Profª Drª Andrea Schwertner Charão}
\institute[UFSM]{Ciência da Computação\\Universidade Federal de Santa Maria}
\date{16/10/2015}

\pgfdeclareimage[height=0.75cm]{university-logo}{images/CienciaDaComputacao.png}
\logo{\pgfuseimage{university-logo}}

% Delete this, if you do not want the table of contents to pop up at
% the beginning of each subsection:
\AtBeginSubsection[]
{
  \begin{frame}<beamer>{Outline}
    \tableofcontents[currentsection,currentsubsection]
  \end{frame}
}


% If you wish to uncover everything in a step-wise fashion, uncomment
% the following command:

%\beamerdefaultoverlayspecification{<+->}


\begin{document}

\begin{frame}
	\titlepage
	%\pgfuseimage{university-logo}
\end{frame}

\begin{frame}{Outline}
  \tableofcontents
\end{frame}

\section{Introdução}
\begin{frame}{Introdução}
	\begin{block}{Teste de software}
	    É uma atividade destrutível, pois visa expor os defeitos para depois corrigir os mesmos.
	\end{block}
	\begin{itemize}
		\item Responsável por apresentar os erros existentes em um determinado programa.
		\item Aumenta a confiança de que o software desempenha as funções especificadas.
	\end{itemize}
\end{frame}

\subsection{Objetivos}
\begin{frame}{Objetivos}
    \begin{block}{Objetivo principal}
	    Apresentar solução de teste reutilizável para novos casos de teste ou novos projetos.
	\end{block}
	\begin{itemize}
		\item Reduzir o trabalho na criação de novos códigos de teste
		\item Tornar solução genérica e replicável
	\end{itemize}
\end{frame}

\subsection{Justificativa}
\begin{frame}{Justificativa}
	\begin{itemize}
		\item Entregar software com maior qualidade
		\item Equipes com recursos humanos limitados
		\item Garantir maior confiabilidade e redução de erros
	\end{itemize}
\end{frame}
\pgfdeclareimage[height=1.0cm]{selenium_logo}{images/selenium_logo.png}
\pgfdeclareimage[height=1.0cm]{cucumber_logo}{images/cucumber_logo.png}
\pgfdeclareimage[height=2.0cm]{junit_logo}{images/junit_logo.jpg}
\pgfdeclareimage[height=6.0cm]{selenium_ide}{images/selenium_ide.PNG}
\section{Fundamentação}
\subsection{Ferramentas de teste de software}
\begin{frame}{Ferramentas de teste de software}
    \pgfuseimage{selenium_logo}
    \item Selenium HQ
	\begin{itemize}
		\item Ferramenta para testes em aplicações web pelo browser de forma automatizada
        \item Os testes rodam diretamente num browser simulando ações de um usuário
        \item Testa se a página web produz o resultado esperado
	\end{itemize}
\end{frame}
\begin{frame}{Ferramentas de teste de software}
    \pgfuseimage{selenium_logo}
    \item Selenium IDE
    \begin{center}\pgfuseimage{selenium_ide}\end{center}
\end{frame}
\begin{frame}[fragile]{Ferramentas de teste de software}
    \item Selenium WebDriver
    \begin{lstlisting}
public class LoginTeste {
    private WebDriver driver;
    private static String url = "inf.ufsm.br/piec";
    
    @Before
    public void setUp() throws Exception { 
        driver = new FirefoxDriver(); 
    }
    
    @Test
    public void testeLoginSucesso() {
        driver.get(url);
        driver.findElement(By.id("login")).sendKeys("lucas");
        driver.findElement(By.name("senha")).sendKeys("senha123");
        driver.findElement(By.cssSelector("button.btn.btn-default")).click();
        assertEquals("Sucesso!", driver.findElement(By.cssSelector("h4")).getText());
    }
    
    @After
    public void tearDown() { driver.quit(); }
}
	\end{lstlisting}
\end{frame}
\begin{frame}[fragile]{Ferramentas de teste de software}
    \pgfuseimage{cucumber_logo}
    \item Cucumber
	\begin{itemize}
		\item Linguagem bem próxima da linguagem natural
		\item Permite escrever cenários que ilustrem as regras de negócio
        \item Serve como documentação das funcionalidades solicitadas
	\end{itemize}
\end{frame}
\begin{frame}[fragile]{Ferramentas de teste de software}
    \pgfuseimage{cucumber_logo}
    \item Cucumber
    \begin{lstlisting}
Funcionalidade: Fazer login
  Contexto: Login com sucesso
    Dado que eu queira acessar o endereço "inf.ufsm.br/login.htm"
    Quando preencho o campo "login" com o valor "lamaral" buscando pelo "id"
    E preencho o campo "senha" com o valor "teste123" buscando pelo "id"
    E clico no elemento "button.btn.btn-default" buscando pelo "css"
    Entao o atributo "Sucesso!" do elemento "box_s" buscando pelo "id" não está nulo
	\end{lstlisting}
\end{frame}
\begin{frame}[fragile]{Ferramentas de teste de software}
    \pgfuseimage{cucumber_logo}
    \item Cucumber
    \begin{lstlisting}
@Dado("^que eu queira acessar o endereco (.*)$")
public void acessarEndereco(String url) {
    // descrever ações aqui
}

@E("^preencho o campo (.*) com o valor (.*) buscando pelo (.*)$")
public void preencherCampo(String campo, String valor, String identificador) {
    // descrever ações aqui
}

@E("^clico no elemento (.*) buscando pelo (.*)$")
public void clicarElemento(String campo, String identificador) {
    // descrever ações aqui
}

@Entao("^comparar se atributo (.*) do elemento (.*) buscando pelo (.*) é nulo$")
public void compararSeNulo(String atributo, String campo, String identificador) {
    assertNull(...);
}

	\end{lstlisting}
\end{frame}
\begin{frame}[fragile]{Ferramentas de teste de software}
    \pgfuseimage{junit_logo}
    \item JUnit
	\begin{itemize}
		\item Usa asserções para testar os resultados esperados
        \item Adequado para os testes unitários e de integração
	\end{itemize}
\end{frame}
\section{Desenvolvimento}
\subsection{Delimitação do escopo}
\begin{frame}{Delimitação do escopo}
	\begin{itemize}
		\item Sistemas web
		\begin{itemize}
			\item Desenvolvidos em Java
			\item Suporte para funções Ajax e Javascript
		\end{itemize}
	\end{itemize}
\end{frame}
\pgfdeclareimage[height=4.0cm]{solucao1}{images/solucao1.png}
\subsection{Visão geral da solução}
\begin{frame}{Visão geral da solução}
    \item Modelagem da solução
	\begin{center}\pgfuseimage{solucao1}\end{center}
\end{frame}
\begin{frame}[fragile]{Visão geral da solução}
    \item Interface de anotação
    \begin{lstlisting}
@Documented
@Retention(RetentionPolicy.RUNTIME)
@Target({ElementType.TYPE, ElementType.METHOD})
public @interface Teste {
    String getUrl() default "";

    //findElement
    String getCampo() default ""; //campo html do formulário
    String getIdentificador() default "id"; //Informar se deve buscar um id, name, etc

    boolean isSelect() default false;

    String getValor() default ""; //utilizado como sendKeys e selectText
    boolean click() default false;
    boolean submit() default false;
    boolean limpar() default false;

    String getTipoAssert() default "igual";
    String getCampoAssert() default "";
    String getIdentificadorAssert() default "id";
    String getValorEsperadoAssert() default "";
    String getAtributoCampoComparacaoAssert() default "texto";
}
	\end{lstlisting}
\end{frame}
\begin{frame}[fragile]{Visão geral da solução}
    \item Exemplo de classe contendo anotações
    \begin{lstlisting}
@Teste(getUrl = "/cadastro-disciplina.htm", getCampo = "salvar", click = true
    ,getIdentificadorAssert = TestePropriedades.IDENTIFICADOR_CSS
    , getCampoAssert = "h4", getValorEsperadoAssert = "Sucesso!")
public class Disciplina {
    private String codigo;
    private String nome;
    private Integer cargaHoraria;

    @Teste(getCampo = "ativa1", click = true)
    public Boolean getAtiva() {
        return ativa == null || ativa;
    }

    public void setAtiva(Boolean ativa) { this.ativa = ativa; }

    @Teste(getCampo = "cargaHoraria", getValor = "60", isSelect = true)
    public Integer getCargaHoraria() { return cargaHoraria; }

    public void setCargaHoraria(Integer cargaHoraria) { 
        this.cargaHoraria = cargaHoraria; 
    }

    @Teste(getCampo = "nome", getValor = "Disciplina teste")
    public String getNome() { return nome; }
}
	\end{lstlisting}
\end{frame}
\begin{frame}[fragile]{Visão geral da solução}
    \item Classe genérica de teste
    \begin{lstlisting}
@Test
public void testaFormularios() {
    for (Class classe : getCarregaClasses()) {
        Teste testeClasse = TestePropriedades.teste(classe);
        if (!testeClasse.getUrl().equals("")) {
            webDriver.get(TestePropriedades.urlSistema + testeClasse.getUrl());
        }
        for(Method metodo: classe.getDeclaredMethods()) {
            Teste teste = TestePropriedades.teste(metodo);
            if (teste != null) {
                executaTeste(teste, true);
            }
        }
        executaTeste(testeClasse, false);
        System.out.println("Formulário da classe "+classe.getName()+" testado!");
    }
}
	\end{lstlisting}
\end{frame}
\subsection{Discussão sobre a solução}
\begin{frame}{Discussão sobre a solução}
     \begin{columns}[T] % contents are top vertically aligned
     \begin{column}[T]{5cm} % each column can also be its own environment
        \begin{block}{Qualidades}
    	    \begin{itemize}
        		\item Suporta testes unitários
        		\item Curva de aprendizado pequena, necessário apenas inserir anotações
        		\item Suporte para funções básicas executadas em um sistema web (click's, seleções, etc)
        	\end{itemize}
    	\end{block}
     \end{column}
     \begin{column}[T]{5cm} % alternative top-align that's better for graphics
        \begin{alertblock}{Limitações}
            \begin{itemize}
            	\item Não possibilita a realização de testes funcionais
            	\item Um campo só pode conter um valor de teste
            	\item Não suporta elementos que dependam de funções Ajax
            \end{itemize}
        \end{alertblock}
     \end{column}
     \end{columns}
\end{frame}
\section{Próximos Passos}
\subsection{Próximos Passos}
\begin{frame}{Próximos Passos}
    \item Permitir testes funcionais
	\begin{itemize}
		\item Acoplar a solução a utilização da linguagem Cucumber, onde cada cenário descreverá um conjunto de ações do Selenium.
	\end{itemize}
    \item Ampliar área de abrangência da solução
	\begin{itemize}
		\item Implementar e disponibilizar suporte para elementos mais complexos.
	\end{itemize}
\end{frame}
\begin{frame}{Próximos Passos}
    \item Validação da solução desenvolvida
	\begin{itemize}
		\item Realizar comparações para validar solução genérica desenvolvida.
		\item Apresentar motivos pelo qual deve-se utilizar a solução e o ganho que a mesma trás para uma equipe que possui recursos humanos limitado.
	\end{itemize}
\end{frame}


\begin{frame}
\titlepage
\end{frame}
\end{document}
\bibliography{../referências}