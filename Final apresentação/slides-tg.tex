% !TEX TS-program = XeLaTeX
% !TEX encoding = UTF-8 Unicode

\documentclass{beamer}

\mode<presentation>
{
  \usetheme{Dresden}

  \setbeamercovered{transparent}
  \useinnertheme{circles}
  \usecolortheme{lsc}
  \beamertemplatenavigationsymbolsempty
}

\setbeamertemplate{headline}{}
\setbeamercovered{invisible}


\usepackage[brazil]{babel}
% or whatever

%\usepackage[utf8]{inputenc}

%\usefonttheme{serif}

\usepackage[T1]{fontenc}
\usepackage{fontspec}
%\setmainfont{Linux Libertine G}
%\setsansfont{Source Sans Pro}
%\setmonofont{Source Code Pro}

\usepackage{textcomp}
\usepackage{listings}

\lstdefinelanguage{Java}{
  keywords={public, private, new, true, boolean, false, catch, void, return, null, for, switch, var, if, in, while, else, case, break, class,
  	Funcionalidade, Contexto, Dado, Quando, E, Entao, Cenario, Delineacao, Exemplos},
  keywordstyle=\color[rgb]{0.2,0.4,0.55}\bfseries,
  ndkeywords={export, throw, implements, import, this,
    @Before, @Test, @After, @Dado, @Quando, @E, @Entao, @Documented, @Retention, @Target, @Teste, @aceitacao, @rejeicao, codigoDis, nomeNovaDisciplina, relevancia, siglaInst, msgErro},
  %ndkeywords={Add, Num},
  ndkeywordstyle=\color[RGB]{218,202,66}\bfseries,
  %identifierstyle=\color{black},
  sensitive=true,
  comment=[l]{//},
  morecomment=[s]{/*}{*/},
  commentstyle=\color{purple}\ttfamily,
 % stringstyle=\color[rgb]{0.0,0.4,0.65}\ttfamily,
  stringstyle=\color[RGB]{34,128,24}\ttfamily,
  %morestring=[b]',
  morestring=[b]"
}

\lstset{
	language=Java,
	basicstyle=\tiny,
	commentstyle=\color[rgb]{0,0.2,0}\normalfont,
	frame=single,
	%texcl=true,
	numbers=left,
	showstringspaces=false,
}

\def\Cpp{{C\nolinebreak[4]\raisebox{.2ex}{\scriptsize\bf++}}}
%\def\Cpp{{C\nolinebreak[4]\raisebox{.3ex}{\small++}}}

\title{Desenvolvimento e Reutilização de Testes Automatizados em Aplicações Web}
\author[Lucas Antunes Amaral]{Lucas Antunes Amaral\\Orientador: Profª Drª Andrea Schwertner Charão}
\institute[UFSM]{Ciência da Computação\\Universidade Federal de Santa Maria}
\date{04/12/2015}

\pgfdeclareimage[height=0.75cm]{university-logo}{images/CienciaDaComputacao.png}
\logo{\pgfuseimage{university-logo}}

% Delete this, if you do not want the table of contents to pop up at
% the beginning of each subsection:
\AtBeginSubsection[]
{
  \begin{frame}<beamer>{Outline}
    \tableofcontents[currentsection,currentsubsection]
  \end{frame}
}


% If you wish to uncover everything in a step-wise fashion, uncomment
% the following command:

%\beamerdefaultoverlayspecification{<+->}


\begin{document}

\begin{frame}
	\titlepage
	%\pgfuseimage{university-logo}
\end{frame}

\begin{frame}{Outline}
  \tableofcontents
\end{frame}

\section{Introdução}
\begin{frame}{Introdução}
	\begin{block}{Teste de software}

	\end{block}
	\begin{itemize}
		\item Responsável por apresentar os erros existentes em um determinado programa
		\item Aumenta a confiança de que o software desempenha as funções especificadas
	\end{itemize}
\end{frame}

\subsection{Objetivos}
\begin{frame}{Objetivos}
    \begin{block}{Objetivo principal}
	    Apresentar solução de teste reutilizável para novos casos de teste ou novos projetos.
	\end{block}
	\begin{itemize}
		\item Reduzir o trabalho na criação de novos códigos de teste
		\item Tornar solução genérica e replicável
	\end{itemize}
\end{frame}

\subsection{Justificativa}
\begin{frame}{Justificativa}
	\begin{itemize}
		\item Entregar software com maior qualidade
		\item Equipes com recursos humanos limitados
		\item Garantir maior confiabilidade e redução de erros
	\end{itemize}
\end{frame}
\pgfdeclareimage[height=1.0cm]{selenium_logo}{images/selenium_logo.png}
\pgfdeclareimage[height=1.0cm]{cucumber_logo}{images/cucumber_logo.png}
\pgfdeclareimage[height=2.0cm]{junit_logo}{images/junit_logo.jpg}
\pgfdeclareimage[height=6.0cm]{selenium_ide}{images/selenium_ide.PNG}
\section{Fundamentação}
\subsection{Ferramentas de teste de software}
\begin{frame}{Ferramentas de teste de software}
    \pgfuseimage{selenium_logo}
    Selenium HQ
	\begin{itemize}
		\item Ferramenta para testes em aplicações web pelo browser de forma automatizada
        \item Os testes rodam diretamente num browser simulando ações de um usuário
        \item Testa se a página web produz o resultado esperado
	\end{itemize}
\end{frame}
\begin{frame}{Ferramentas de teste de software}
    \pgfuseimage{selenium_logo}
    Selenium IDE
    \begin{center}\pgfuseimage{selenium_ide}\end{center}
\end{frame}
\begin{frame}[fragile]{Ferramentas de teste de software}
    \item Selenium WebDriver
    \begin{lstlisting}
public class LoginTeste {
    private WebDriver driver;
    private static String url = "inf.ufsm.br/piec";

    @Before
    public void setUp() throws Exception {
        driver = new FirefoxDriver();
    }

    @Test
    public void testeLoginSucesso() {
        driver.get(url);
        driver.findElement(By.id("login")).sendKeys("lucas");
        driver.findElement(By.name("senha")).sendKeys("senha123");
        driver.findElement(By.cssSelector("button.btn.btn-default")).click();
        assertEquals("Sucesso!", driver.findElement(By.cssSelector("h4")).getText());
    }

    @After
    public void tearDown() { driver.quit(); }
}
	\end{lstlisting}
\end{frame}
\begin{frame}[fragile]{Ferramentas de teste de software}
    \pgfuseimage{cucumber_logo}
    \item Cucumber
	\begin{itemize}
		\item Linguagem bem próxima da linguagem natural
		\item Permite escrever cenários que ilustrem as regras de negócio
        \item Serve como documentação das funcionalidades solicitadas
	\end{itemize}
\end{frame}
\begin{frame}[fragile]{Ferramentas de teste de software}
    \pgfuseimage{cucumber_logo}
    \item Cucumber
    \begin{lstlisting}
Funcionalidade: Fazer login
  Contexto: Login com sucesso
    Dado que eu queira acessar o endereço "inf.ufsm.br/login.htm"
    Quando preencho o campo "login" com o valor "lamaral" buscando pelo "id"
    E preencho o campo "senha" com o valor "teste123" buscando pelo "id"
    E clico no elemento "button.btn.btn-default" buscando pelo "css"
    Entao o atributo "Sucesso!" do elemento "box_s" buscando pelo "id" não está nulo
	\end{lstlisting}
\end{frame}
\begin{frame}[fragile]{Ferramentas de teste de software}
    \pgfuseimage{cucumber_logo}
    \item Cucumber
    \begin{lstlisting}
@Dado("^que eu queira acessar o endereco (.*)$")
public void acessarEndereco(String url) {
    // descrever ações aqui
}

@E("^preencho o campo (.*) com o valor (.*) buscando pelo (.*)$")
public void preencherCampo(String campo, String valor, String identificador) {
    // descrever ações aqui
}

@E("^clico no elemento (.*) buscando pelo (.*)$")
public void clicarElemento(String campo, String identificador) {
    // descrever ações aqui
}

@Entao("^comparar se atributo (.*) do elemento (.*) buscando pelo (.*) é nulo$")
public void compararSeNulo(String atributo, String campo, String identificador) {
    assertNull(...);
}

	\end{lstlisting}
\end{frame}
%\begin{frame}[fragile]{Ferramentas de teste de software}
%    \pgfuseimage{junit_logo}
%    \item JUnit
	%\begin{itemize}
	%	\item Usa asserções para testar os resultados esperados
     %   \item Adequado para os testes unitários e de integração
	%\end{itemize}
%\end{frame}
\section{Desenvolvimento}
\subsection{Delimitação do escopo}
\begin{frame}{Delimitação do escopo}
	\begin{itemize}
		\item Sistemas web
		\begin{itemize}
			\item Desenvolvidos em Java
			\item Suporte para funções Ajax e Javascript
		\end{itemize}
	\end{itemize}
\end{frame}
\pgfdeclareimage[height=4.0cm]{solucao1}{images/solucao1.png}
\subsection{Solução para testes unitários}
\begin{frame}{Solução para testes unitários}
    Modelagem da solução utilizando anotações e Selenium
	\begin{center}\pgfuseimage{solucao1}\end{center}
\end{frame}
\begin{frame}[fragile]{Solução para testes unitários}
    Interface de anotação
    \begin{lstlisting}
@Documented
@Retention(RetentionPolicy.RUNTIME)
@Target({ElementType.TYPE, ElementType.METHOD})
public @interface Teste {
    String getUrl() default "";

    //findElement
    String getCampo() default ""; //campo html do formulário
    String getIdentificador() default "id"; //Informar se deve buscar um id, name, etc

    boolean isSelect() default false;

    String getValor() default ""; //utilizado como sendKeys e selectText
    boolean click() default false;
    boolean submit() default false;
    boolean limpar() default false;

    String getTipoAssert() default "igual";
    String getCampoAssert() default "";
    String getIdentificadorAssert() default "id";
    String getValorEsperadoAssert() default "";
    String getAtributoCampoComparacaoAssert() default "texto";
}
	\end{lstlisting}
\end{frame}
\begin{frame}[fragile]{Solução para testes unitários}
    Exemplo de classe contendo anotações
    \begin{lstlisting}
@Teste(getUrl = "/cadastro-disciplina.htm", getCampo = "salvar", click = true
    ,getIdentificadorAssert = TestePropriedades.IDENTIFICADOR_CSS
    , getCampoAssert = "h4", getValorEsperadoAssert = "Sucesso!")
public class Disciplina {
    private String codigo;
    private String nome;
    private Integer cargaHoraria;

    @Teste(getCampo = "ativa1", click = true)
    public Boolean getAtiva() {
        return ativa == null || ativa;
    }

    public void setAtiva(Boolean ativa) { this.ativa = ativa; }

    @Teste(getCampo = "cargaHoraria", getValor = "60", isSelect = true)
    public Integer getCargaHoraria() { return cargaHoraria; }

    public void setCargaHoraria(Integer cargaHoraria) {
        this.cargaHoraria = cargaHoraria;
    }

    @Teste(getCampo = "nome", getValor = "Disciplina teste")
    public String getNome() { return nome; }
}
	\end{lstlisting}
\end{frame}
\begin{frame}[fragile]{Solução para testes unitários}
    Classe genérica de teste
    \begin{lstlisting}
@Test
public void testaFormularios() {
    for (Class classe : getCarregaClasses()) {
        Teste testeClasse = TestePropriedades.teste(classe);
        if (!testeClasse.getUrl().equals("")) {
            webDriver.get(TestePropriedades.urlSistema + testeClasse.getUrl());
        }
        for(Method metodo: classe.getDeclaredMethods()) {
            Teste teste = TestePropriedades.teste(metodo);
            if (teste != null) {
                executaTeste(teste, true);
            }
        }
        executaTeste(testeClasse, false);
        System.out.println("Formulário da classe "+classe.getName()+" testado!");
    }
}
	\end{lstlisting}
\end{frame}
%\subsubsection{Discussão sobre a solução para testes unitários}
\begin{frame}{Discussão sobre a solução para testes unitários}
     \begin{columns}[T] % contents are top vertically aligned
     \begin{column}[T]{5cm} % each column can also be its own environment
        \begin{block}{Qualidades}
    	    \begin{itemize}
        		\item Suporta testes unitários
        		\item Curva de aprendizado pequena, necessário apenas inserir anotações
        		\item Suporte para funções básicas executadas em um sistema web (click's, seleções, etc)
        	\end{itemize}
    	\end{block}
     \end{column}
     \begin{column}[T]{5cm} % alternative top-align that's better for graphics
        \begin{alertblock}{Limitações}
            \begin{itemize}
            	\item Não possibilita a realização de testes funcionais
            	\item Um campo só pode conter um valor de teste
            	\item Não suporta elementos que dependam de funções Ajax
            \end{itemize}
        \end{alertblock}
     \end{column}
     \end{columns}
\end{frame}
\pgfdeclareimage[height=6.0cm]{solucao2}{images/solucao2.png}
\subsection{Solução para testes funcionais utilizando Cucumber}
\begin{frame}{Solução para testes funcionais utilizando Cucumber}
    Modelagem da solução utilizando Cucumber e Selenium WebDriver
	\begin{center}\pgfuseimage{solucao2}\end{center}
\end{frame}
\begin{frame}[fragile]{Solução para testes funcionais utilizando Cucumber}
    \item Classe genérica de teste
    \begin{lstlisting}
public Object getValorPropriedadeCampo(WebElement webElement, String atributo) {
    if (atributo.equals(ATRIBUTO_COMPARACAO_ASSERT_TEXTO)) {
        return webElement.getText();
    } else if (atributo.equals(ATRIBUTO_COMPARACAO_ASSERT_NOME_TAG)) {
        return webElement.getTagName();
    } else if ...
} //Retorna atributo de um campo

//Retorna um elemento HTML
public WebElement getEncontraCampo(String identificador, String campo) {
    return webDriver.findElement(getEncontra(identificador, campo));
}

public By getEncontra(String identificador, String campo) {
    if (identificador.equals(IDENTIFICADOR_ID)) {
        return By.id(campo);
    } else if ...
}
	\end{lstlisting}
\end{frame}
\begin{frame}[fragile]{Solução para testes funcionais utilizando Cucumber}
    \item Classe que descreve steps
    \begin{lstlisting}
@Quando("^seleciono a opcao (.*) no campo (.*) buscando pelo (.*)$")
public void selecionarOpcaoCampo(String opcao, String campo, String identificador) {
    WebElement webElement = getEncontraCampo(identificador, campo);
    Select select = new Select(webElement);
    select.selectByVisibleText(opcao);
}

 @E("^preencho o campo (.*) com o valor (.*) buscando pelo (.*)$")
public void preencherCampo(String campo, String valor, String identificador) {
    WebElement webElement = getEncontraCampo(identificador, campo);
    webElement.sendKeys(valor);
}

@E("^aguardo (\\d+) milisegundos para verificar se elemento (.*) buscando pelo (.*)
    esta visivel$")
public void aguardarCampoVisivel(Long tempo, String campo, String identificador) {
    WebDriverWait webDriverWait = new WebDriverWait(webDriver, tempo);
    webDriverWait.until(ExpectedConditions.visibilityOf(funcoes.
        getEncontraCampo(identificador, campo)));
}
    \end{lstlisting}
\end{frame}
\begin{frame}[fragile]{Solução para testes funcionais utilizando Cucumber}
    \item Escrevendo cenários com Cucumber
    \begin{lstlisting}
Funcionalidade: Ações executadas por um aluno do curso de Ciência da Computação no
registro de seu plano individual de estudos complementares.

  Contexto: Fazer login no sistema como um aluno
    Dado acesso o endereco "/login.htm"
    Quando preencho o campo "login" com o valor "lamaral" buscando pelo "id"
    E preencho o campo "senha" com o valor "teste123" buscando pelo "id"
    E clico no elemento "button.btn.btn-default" buscando pelo "css"
    Entao verifico se atributo "nome tag" do elemento "id" buscando pelo "nome"
        não está nulo

  Cenario: Adicionar umas disciplina ao plano com sucesso.
    Dado seleciono a opcao "ELC1051" no campo "idDiscpAdicionar" buscando pelo "id"
    Quando preencho o campo "curso" com o valor "C. da Computação" buscando pelo "id"
    E preencho o campo "semestreRealizacao" com o valor "II/2011" buscando pelo "id"
    E clico no elemento "adicionarPiecDisciplina" buscando pelo "id"
    Entao comparo a igualdade entre o valor esperado "Sucesso!" com atributo "texto"
    do elemento "h4" buscando pelo "css"
    \end{lstlisting}
\end{frame}
\begin{frame}{Discussão sobre a solução para testes funcionais utilizando Cucumber}
     \begin{columns}[T] % contents are top vertically aligned
     \begin{column}[T]{5cm} % each column can also be its own environment
        \begin{block}{Qualidades}
    	    \begin{itemize}
    	        \item Permite realizar testes funcionais
        		\item Reutilização dos \emph{steps} definidos
        		\item Cenários com escrita de fácil compreensão (documentação)
        		\item Suporte para funções dependentes de chamadas Ajax e Javascript
        	\end{itemize}
    	\end{block}
     \end{column}
     \begin{column}[T]{5cm} % alternative top-align that's better for graphics
        \begin{alertblock}{Limitações}
            \begin{itemize}
            	\item Não possibilita realizar testes que dependam de condições (if, else)
            	\item Reutilizar cenário dentro de outro cenário
            	\item Retroceder o BD para estado inicial após os testes
            	\item Somente possibilita manipular tipos de dados básicos (int, float, String)
            \end{itemize}
        \end{alertblock}
     \end{column}
     \end{columns}
\end{frame}
\pgfdeclareimage[height=5.0cm]{solucao3}{images/solucao3.png}
\subsection{Solução para testes funcionais utilizando Selenium}
\begin{frame}{Solução para testes funcionais utilizando Selenium}
    Modelagem da solução utilizando Selenium WebDriver
	\begin{center}\pgfuseimage{solucao3}\end{center}
\end{frame}
\begin{frame}[fragile]{Solução para testes funcionais utilizando Selenium}
    Escrevendo cenários em Java
    \begin{lstlisting}
private void abreFechaAgrupamento() {
	navegador.clicarElemento("img[alt=\"Visualizar\"]", "css");
	navegador.aguardarCampoVisivel(100l, "img[alt=\"F\"]", "css");
	WebElement dataAbertura = navegador.getEncontraCampo("data_abertura_1");
	WebElement dataFechamento = navegador.getEncontraCampo("data_fechamento_1");
	boolean aberto = true;
	if (dataAbertura.getText() != null) {
		aberto = false;
	}
	navegador.clicarElemento("img[alt=\"F\"]");
	if (aberto) {
		assertEquals(SilasFuncoesDatas.sdf.format(new Date()),
		    dataFechamento.getText());
	} else {
		assertEquals(SilasFuncoesDatas.sdf.format(new Date()),
		    dataAbertura.getText());
	}
}
    \end{lstlisting}
\end{frame}
\begin{frame}{Discussão sobre a solução para testes funcionais utilizando Selenium}
     \begin{columns}[T] % contents are top vertically aligned
     \begin{column}[T]{5cm} % each column can also be its own environment
        \begin{block}{Qualidades}
    	    \begin{itemize}
    	        \item Permite realizar testes funcionais
        		\item Reutilização dos \emph{steps} definidos
        		\item Reutilizar cenário dentro de outro cenário
        		\item Possibilita realizar testes que dependam de condições (if, else)
        		\item Permite manipular todos tipos de dados disponíveis em Java
        	\end{itemize}
    	\end{block}
     \end{column}
     \begin{column}[T]{5cm} % alternative top-align that's better for graphics
        \begin{alertblock}{Limitações}
            \begin{itemize}
            	\item Maior complexidade na compreensão do cenário
            	\item Retroceder o BD para estado inicial após os testes
            	\item Digita-se mais em comparação com a solução anterior
            \end{itemize}
        \end{alertblock}
     \end{column}
     \end{columns}
\end{frame}

\section{Resultados}
\subsection{Resultados}
\begin{frame}{Resultados}
    Vantagens e desvantagens das abordagens de teste confeccionadas
    \begin{table}[!htpb]
        \tiny
        \begin{tabular}{p{5cm}|c|c|c}
        	& Unitário & Cucumber e Selenium & Funcional Selenium\\ \hline
        	Testes unitários & \checkmark & \checkmark & \checkmark \\ \hline
        	Testes funcionais &  & \checkmark & \checkmark \\ \hline
        	Elementos que dependam de funções Ajax e Javascript & & \checkmark & \checkmark \\ \hline
        	Cenários com escrita de fácil compreensão & & \checkmark & \\ \hline
        	Realização de testes dependentes de condições & & & \checkmark \\ \hline
        	Armazenar valor de um campo para utiliza-lo posteriormente & & & \checkmark \\ \hline
        	Possibilita utilizar parâmetros que diferem dos tipos básicos (int, float, String) & \checkmark & & \checkmark \\ \hline
        	Retroceder o banco de dados ao último estado válido após realizar os testes & & & \\ \hline
        	Utilização de cenários dentro de outro cenário & & & \checkmark \\ \hline
        \end{tabular}
    \end{table}
\end{frame}
\begin{frame}{Resultados}
    Comparação com teste descrito somente com ferramenta para teste web
    \begin{lstlisting}
public void cadastrarDisciplinaSucesso() {
	WebDriver webDriver = new FirefoxDriver();
	webDriver.get(url + "/cadastro-disciplina.htm");
	webDriver.findElement(By.id("codigo")).sendKeys("ELC9898");
	webDriver.findElement(By.id("nome")).sendKeys("Disciplina nova");
	Select cargaHoraria = new Select(webDriver.findElement(By.id("cargaHoraria")));
	cargaHoraria.selectByVisibleText("60");
	webDriver.findElement(By.id("ativa1")).click();
	Select instituicao = new Select(webDriver.findElement(By.id("idInstituicao")));
	instituicao.selectByVisibleText("UFSM - Universidade Federal de SM");
	webDriver.findElement(By.id("preAprovada1")).click();
	webDriver.findElement(By.id("salvar")).click();
	assertEquals("Sucesso!", webDriver.findElement(By.cssSelector("h4")).getText());
}
    \end{lstlisting}
\end{frame}
\begin{frame}
	\titlepage
\end{frame}
\end{document}